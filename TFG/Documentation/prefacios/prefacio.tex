\thispagestyle{empty}

\begin{center}
{\large\bfseries ChiselVR \\ Aplicación para la creación de esculturas en entornos de realidad virtual }\\
\end{center}
\begin{center}
Jordi Conde Molina\\
\end{center}

%\vspace{0.7cm}

\vspace{0.5cm}
\noindent\textbf{Palabras clave}: \textit{software libre, informática gráfica, realidad virtual, escultura, simulación, Unreal Engine 5, Geometry Script, operaciones booleanas}
\vspace{0.7cm}

\noindent\textbf{Resumen}\\
La realidad virtual nos ofrece un sinfín de oportunidades para aplicaciones que simplemente son imposibles sin esta. Medicina, aviación o entretenimiento son solo algunas de las industrias en las que la realidad virtual trae nuevas formas de experimentar, investigar y disfrutar, ya sea con videojuegos con mecánicas nunca antes vistas o simuladores donde lo único que separa realidad de ficción es la calidad de los gráficos, una barrera que es cada vez más fina.

Y es que de hecho, en los últimos años, los avances en potencia gráfica y en el \textit{hardware} de realidad virtual hacen que estas opciones sean más interesantes que nunca, alcanzando una experiencia más cómoda que nunca para el usuario y abriendo la puerta a aplicaciones que en el pasado simplemente hubiesen sido imposibles de hacer desde un punto de vista técnico.

En este proyecto, se propone pues la creación de una aplicación de realidad virtual en la que simular la experiencia real de esculpir sobre mármol, algo hasta ahora poco explorado por el coste computacional que supone algo de este calibre. En esta aplicación, el usuario podrá usar distintas herramientas con las que modificar un bloque de mármol como si de uno real se tratase, y experimentar así el arte de la escultura sin la inversión que supone todo el material que este requiere.

\cleardoublepage

\begin{center}
	{\large\bfseries ChiselVR \\ Application for the creation of sculptures in virtual reality environments}\\
\end{center}
\begin{center}
	Jordi Conde Molina\\
\end{center}
\vspace{0.5cm}
\noindent\textbf{Keywords}: \textit{open source, computer graphics, virtual reality, sculpture, simulation, Unreal Engine 5, Geometry Script, boolean operations.}
\vspace{0.7cm}

\noindent\textbf{Abstract}\\
Virtual reality offers us a myriad of opportunities for applications that are simply impossible without it. Medicine, aviation, and entertainment are just some of the industries in which virtual reality brings new ways to experience, explore, and enjoy, be it through video games with unprecedented mechanics, or simulators where the only thing separating reality from fiction is the quality of graphics, a barrier that is becoming increasingly thinner.

In fact, in recent years, advances in graphics and virtual reality hardware make these options more captivating than ever, achieving a more comfortable experience for users and opening the door to applications that in the past would have been impossible from a technical standpoint.

In this project, the creation of a virtual reality application is proposed, one that simulates the real experience of sculpting on marble – an area that has been relatively unexplored due to the computational cost it entails. Within this application, users will be able to utilize different tools to modify a block of marble as if it were a real one, thus experiencing the art of sculpture without the investment required for all the materials it typically demands.

\chapter*{Agradecimientos}

A mis padres y a Paula por aguantar mis lloros.

A mis amigos por ayudarme a entender cómo se hace un TFG.

A Mosto por dejarme acariciarle en momentos de agobio.




